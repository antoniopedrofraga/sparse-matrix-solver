
%%%%%%%%%%%%%%%%%%%%%%%%%%%%%%%%%%%%%%%%%%%%%%%%%%%%%%%%%%%%%%%%%%%%%
% LaTeX Template: Project Titlepage Modified (v 0.1) by rcx
%
% Original Source: http://www.howtotex.com
% Date: February 2014
% 
% This is a title page template which be used for articles & reports.
% 
% This is the modified version of the original Latex template from
% aforementioned website.
% 
%%%%%%%%%%%%%%%%%%%%%%%%%%%%%%%%%%%%%%%%%%%%%%%%%%%%%%%%%%%%%%%%%%%%%%

\documentclass[12pt]{article}
\usepackage[utf8]{inputenc}
\usepackage[a4paper]{geometry}
\usepackage[myheadings]{fullpage}
\usepackage{enumitem}
\usepackage{fancyhdr}
\usepackage{lastpage}
\usepackage{graphicx, wrapfig, subcaption, setspace, booktabs}
\usepackage[T1]{fontenc}
\usepackage[font=small, labelfont=bf]{caption}
\usepackage{fourier}
\usepackage{amsmath}
\usepackage[protrusion=true, expansion=true]{microtype}
\usepackage[english]{babel}
\usepackage{sectsty}
\usepackage{url, lipsum}
\usepackage{titlesec}
\usepackage{diagbox}
\usepackage{pdfpages}

\usepackage{listings}
\usepackage{color}

\definecolor{dkgreen}{rgb}{0,0.6,0}
\definecolor{gray}{rgb}{0.5,0.5,0.5}
\definecolor{mauve}{rgb}{0.58,0,0.82}

\lstset{frame=tb,
  language=C++,
  aboveskip=3mm,
  belowskip=3mm,
  showstringspaces=false,
  columns=flexible,
  basicstyle={\small\ttfamily},
  numbers=none,
  numberstyle=\tiny\color{gray},
  keywordstyle=\color{blue},
  commentstyle=\color{dkgreen},
  stringstyle=\color{mauve},
  breakatwhitespace=true,
  breaklines=true,
  tabsize=2
}

\newcommand{\HRule}[1]{\rule{\linewidth}{#1}}
\onehalfspacing
\setcounter{tocdepth}{5}
\setcounter{secnumdepth}{5}
\inputencoding{utf8}

\titleformat{\paragraph}
{\normalfont\normalsize\bfseries}{\theparagraph}{1em}{}
\titlespacing*{\paragraph}
{0pt}{3.25ex plus 1ex minus .2ex}{1.5ex plus .2ex}

%-------------------------------------------------------------------------------
% HEADER & FOOTER
%-------------------------------------------------------------------------------
\pagestyle{fancy}
\fancyhf{}
\setlength\headheight{15pt}
\fancyhead[L]{António Pedro Araújo Fraga}
\fancyhead[R]{Cranfield University}
\fancyfoot[R]{Page \thepage\ of \pageref{LastPage}}
%-------------------------------------------------------------------------------
% TITLE PAGE
%-------------------------------------------------------------------------------

\begin{document}

\title{ \fontsize{40}{90} \textsc{Small Scale for Parallel Programming}
		\\ [2.0cm]
		\HRule{0.5pt} \\
		\LARGE \textbf{Sparse Matrix-Vector Product Kernel}
		\HRule{2pt} \\ [0.5cm]
		\normalsize \today \vspace*{5\baselineskip}}

\date{}

\author{
		\textbf{António Pedro Araújo Fraga} \\
		\textbf{Student ID: 279654} \\ 
		\textbf{Cranfield University} \\
		\textbf{M.Sc. in Software Engineering for Technical Computing
		} }

\maketitle
\thispagestyle{empty}
\newpage
\tableofcontents
\thispagestyle{empty}
\newpage
\null\vspace{\fill}
\begin{abstract}
\normalsize
The product of a sparse matrix and a vector was calculated both in parallel and sequentially. The parallel procedure was executed with two different technologies, Open Multi-Processing and Compute Unified Device Architecture. Sparse matrices were stored in two different formats, Compressed Sparse Row and Ellpack. Different procedures produced different effects, those effects were discussed and studied. 
\end{abstract}
\vspace{\fill}
\thispagestyle{empty}
\newpage

%-------------------------------------------------------------------------------
% Section title formatting
\sectionfont{\scshape}
\titleformat{\section}
{\normalfont\huge\bfseries}{\thesection}{1em}{}
\titleformat{\subsection}
{\normalfont\large\bfseries}{\thesubsection}{1em}{}
\titlespacing*{\section}
{0pt}{5.5ex plus 1ex minus .2ex}{4.3ex plus .2ex}
\titlespacing*{\subsection}
{0pt}{5.5ex plus 1ex minus .2ex}{4.3ex plus .2ex}
%-------------------------------------------------------------------------------

%-------------------------------------------------------------------------------
% BODY
%-------------------------------------------------------------------------------

%-------------------------------------------------------------------------------
% Nomenclature
%-------------------------------------------------------------------------------
\begin{table}[tb]
\caption{Nomenclature}
\label{tab:notation}
\centering
\def\arraystretch{1.5}
\begin{tabular}{ll}
Matrix Rows & $m$\\
Matrix Columns & $n$\\
\end{tabular}
\end{table}

%-------------------------------------------------------------------------------
% Introduction
%-------------------------------------------------------------------------------

\section*{Introduction}
\addcontentsline{toc}{section}{Introduction}

\par Bidimensional matrices are often represented in a bidimensional array of values of \textbf{m} by \textbf{n} elements.


\subsection*{Problem definition}
\addcontentsline{toc}{subsection}{Problem definition}


\subsection*{Numerical analysis}
\addcontentsline{toc}{subsection}{Numerical analysis}


\section*{Procedures}
\addcontentsline{toc}{section}{Procedures}


\section*{Solution Design}
\addcontentsline{toc}{section}{Solution Design}


\section*{Results \& Discussion}
\addcontentsline{toc}{section}{Results \& Discussion}


\pagebreak
\section*{Conclusions}
\addcontentsline{toc}{section}{Conclusions}

 

%-------------------------------------------------------------------------------
% REFERENCES
%-------------------------------------------------------------------------------
\newpage
%\addcontentsline{toc}{section}{References}
\begin{thebibliography}{0}

\bibitem{sparse}
Raphael Yuster and Uri Zwick, \textit{Fast sparse matrix multiplication}, Available at: <\url{http://www.cs.tau.ac.il/~zwick/papers/sparse.pdf}> [Accessed 28 March 2017]

\bibitem{sparse-gpu}
B. Neelima1 and Prakash S. Raghavendra, April 2012,  \textit{Effective Sparse Matrix Representation for the
GPU Architectures}, Available at: <\url{https://pdfs.semanticscholar.org/2d15/dd5d0975fff797397ad31059ec097b659e00.pdf}> [Accessed 28 March 2017]

\end{thebibliography}
\newpage

\section*{Appendices}
\addcontentsline{toc}{section}{Appendices}


\newpage

\subsection*{Source Code}
\addcontentsline{toc}{subsection}{Source Code}


\addcontentsline{toc}{subsection}{Doxygen Documentation}

\end{document}

