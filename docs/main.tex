
%%%%%%%%%%%%%%%%%%%%%%%%%%%%%%%%%%%%%%%%%%%%%%%%%%%%%%%%%%%%%%%%%%%%%
% LaTeX Template: Project Titlepage Modified (v 0.1) by rcx
%
% Original Source: http://www.howtotex.com
% Date: February 2014
% 
% This is a title page template which be used for articles & reports.
% 
% This is the modified version of the original Latex template from
% aforementioned website.
% 
%%%%%%%%%%%%%%%%%%%%%%%%%%%%%%%%%%%%%%%%%%%%%%%%%%%%%%%%%%%%%%%%%%%%%%

\documentclass[12pt]{article}
\usepackage[utf8]{inputenc}
\usepackage[a4paper]{geometry}
\usepackage[myheadings]{fullpage}
\usepackage{enumitem}
\usepackage{fancyhdr}
\usepackage{lastpage}
\usepackage{graphicx, wrapfig, subcaption, setspace, booktabs}
\usepackage[T1]{fontenc}
\usepackage[font=small, labelfont=bf]{caption}
\usepackage{fourier}
\usepackage{amsmath}
\usepackage[protrusion=true, expansion=true]{microtype}
\usepackage[english]{babel}
\usepackage{sectsty}
\usepackage{url, lipsum}
\usepackage{titlesec}
\usepackage{diagbox}
\usepackage{pdfpages}

\usepackage{listings}
\usepackage{color}

\definecolor{dkgreen}{rgb}{0,0.6,0}
\definecolor{gray}{rgb}{0.5,0.5,0.5}
\definecolor{mauve}{rgb}{0.58,0,0.82}

\lstset{
  language=C++,
  aboveskip=3mm,
  belowskip=3mm,
  showstringspaces=false,
  columns=flexible,
  basicstyle={\small\ttfamily},
  numbers=none,
  numberstyle=\tiny\color{gray},
  keywordstyle=\color{blue},
  commentstyle=\color{dkgreen},
  stringstyle=\color{mauve},
  breakatwhitespace=true,
  breaklines=true,
  tabsize=2,
  xleftmargin=.2\textwidth, xrightmargin=.2\textwidth,
  otherkeywords={1, 2, 3, 4, 5, 6, 7, 8, 9, 0},
  morekeywords=[2]{1, 2, 3, 4, 5, 6, 7, 8, 9, 0},
  keywordstyle=[2]{\color{purple}}
}

\newcommand{\HRule}[1]{\rule{\linewidth}{#1}}
\onehalfspacing
\setcounter{tocdepth}{5}
\setcounter{secnumdepth}{5}
\setcounter{MaxMatrixCols}{20}
\inputencoding{utf8}

\titleformat{\paragraph}
{\normalfont\normalsize\bfseries}{\theparagraph}{1em}{}
\titlespacing*{\paragraph}
{0pt}{3.25ex plus 1ex minus .2ex}{1.5ex plus .2ex}

%-------------------------------------------------------------------------------
% HEADER & FOOTER
%-------------------------------------------------------------------------------
\pagestyle{fancy}
\fancyhf{}
\setlength\headheight{15pt}
\fancyhead[L]{António Pedro Araújo Fraga}
\fancyhead[R]{Cranfield University}
\fancyfoot[R]{Page \thepage\ of \pageref{LastPage}}
%-------------------------------------------------------------------------------
% TITLE PAGE
%-------------------------------------------------------------------------------

\begin{document}

\title{ \fontsize{40}{90} \textsc{Small Scale for Parallel Programming}
		\\ [2.0cm]
		\HRule{0.5pt} \\
		\LARGE \textbf{Sparse Matrix-Vector Product Kernel}
		\HRule{2pt} \\ [0.5cm]
		\normalsize \today \vspace*{5\baselineskip}}

\date{}

\author{
		\textbf{António Pedro Araújo Fraga} \\
		\textbf{Student ID: 279654} \\ 
		\textbf{Cranfield University} \\
		\textbf{M.Sc. in Software Engineering for Technical Computing
		} }

\maketitle
\thispagestyle{empty}
\newpage
\tableofcontents
\thispagestyle{empty}
\newpage
\null\vspace{\fill}
\begin{abstract}
\normalsize
The product of a sparse matrix and a vector was calculated both in parallel and sequentially. The parallel procedure was executed with two different technologies, Open Multi-Processing and Compute Unified Device Architecture. Sparse matrices were stored in two different formats, Compressed Sparse Row and Ellpack. It was seen that the two formats were adequate for different types of matrices. Different procedures produced different effects, those effects were discussed and studied.
\end{abstract}
\vspace{\fill}
\thispagestyle{empty}
\newpage

%-------------------------------------------------------------------------------
% Section title formatting
\sectionfont{\scshape}
\titleformat{\section}
{\normalfont\huge\bfseries}{\thesection}{1em}{}
\titleformat{\subsection}
{\normalfont\large\bfseries}{\thesubsection}{1em}{}
\titlespacing*{\section}
{0pt}{5.5ex plus 1ex minus .2ex}{4.3ex plus .2ex}
\titlespacing*{\subsection}
{0pt}{5.5ex plus 1ex minus .2ex}{4.3ex plus .2ex}
%-------------------------------------------------------------------------------

%-------------------------------------------------------------------------------
% BODY
%-------------------------------------------------------------------------------

%-------------------------------------------------------------------------------
% Nomenclature
%-------------------------------------------------------------------------------
\begin{table}[tb]
\caption{Nomenclature}
\label{tab:notation}
\centering
\def\arraystretch{1.5}
\begin{tabular}{ll}
Matrix number of rows & $m$\\
Matrix number of columns & $n$\\
Sparse matrix & $A$\\
Vector to be multiplied & $x$\\
Resulting vector & $b$\\
Floating point operations per second & \textit{FLOPS} \\
Number of non-zero values & \textit{nz} \\
Maximum number of non-zero values per row & \textit{maxnz} \\
Average computation time of a given kernel & \textit{T} \\
\end{tabular}
\end{table}

%-------------------------------------------------------------------------------
% Introduction
%-------------------------------------------------------------------------------

\section*{Introduction}
\addcontentsline{toc}{section}{Introduction}

\par Bidimensional matrices are often represented in a bidimensional array of values of $m \times n$ elements. When matrices with this representation are multiplied by a vector, one has to iterate through every element of a matrix. This approach can be rather time expensive when dealing with large matrices.
\par Sparse Matrices have a particular characteristic. They are formed by a larger number of \textbf{zero} values, compared to the amount of \textbf{non-zero} elements. Therefore, these matrices can be represented in different formats in order to avoid multiplications by zero.\cite{sparse-gpu} The distribution of \textbf{non-zero} values within these matrices is unpredictable. Hence, one has to come up with workable formats, allowing to have knowledge of a particular row, column and value.

\subsection*{Sparse Matrices Formats}
\addcontentsline{toc}{subsection}{Sparse Matrices Formats}

\par Sparse Matrices can be stored in several formats like \textbf{DIA} (diagonal), more adequate for matrices which elements are spread across the diagonal,  and \textbf{COO} (Coordinate Format).\cite{sparse-gpu} In this project, the \textbf{CSR} (Compressed Sparse Row) and \textbf{Ellpack} formats were the only ones to be compared. Using \textbf{0-based indexing}, and having a matrix defined by,

\begin{center}
\centering
\begin{minipage}{.4\textwidth}
  \flushright
  A  =   
	$
	\begin{pmatrix}
    & \textbf{3} & \textbf{4} & 0 & 0 & 0 & \\
    & 0 & \textbf{5} & \textbf{1} & 0 & 0 & \\
    & 0 & \textbf{11} & \textbf{12} & 0 & 0 & \\
    & 0 & 0 & \textbf{2} & \textbf{13} & 0 & \\
    & 0 & 0 & 0 & \textbf{1} & \textbf{-1} & \\
	\end{pmatrix}
	$
\end{minipage}%
\begin{minipage}{.4\textwidth}
 \center
  \textit{m} = $5$
  \linebreak
  \textit{n} = $5$
\end{minipage}
\linebreak
\end{center}

\par the CSR format can be represented by declaring the number of \textit{non-zero} values, \textit{nz}. An array of values, \textit{as}, of length \textbf{\textit{nz}}. One array of pointers, \textit{irp}, indicating which value in the previous array belongs to the next row. And finally, an array of columns of length \textbf{\textit{nz}}, \textit{ja}, indicating the column index of each value. Therefore, for the given matrix, one can write the described arrays and value as,

\begin{center}
\textit{nz}  =  10 
\linebreak
\linebreak
\textit{irp}  =   
$
\begin{pmatrix}
    & 0 & 2 & 4 & 6 & 8 & \\
\end{pmatrix}
$
\linebreak
\linebreak
\textit{as}  =   
$
\begin{pmatrix}
    & 3 & 4 & 5 & 1 & 11 & 12 & 2 & 13 & 1 & -1 & \\
\end{pmatrix}
$
\linebreak
\linebreak
\textit{ja}  =   
$
\begin{pmatrix}
    & 0 & 1 & 1 & 2 & 1 & 2 & 2 & 3 & 3 & 4 & \\
\end{pmatrix}
$
\linebreak
\end{center}

\par Following a C-like syntax, the $A . x = b$ equation, with \textit{A} following the previously described format, can be solved by,

\begin{lstlisting}
for (int i = 0; i < m; ++i) {
		double temp = 0.0;
		for (int j = irp[i]; j < irp[i + 1]; ++j) {
				temp += as[j] * x[ja[j]];
		}
		y[i] = temp;
}
\end{lstlisting}

\par One can represent a matrix in the \textbf{Ellpack} format by defining two \textbf{bidimensional} arrays and one value. Declaring \textit{maxnz} as the maximum number of non-zero values in a given row, one can declare two arrays of \textit{m} rows by \textit{maxnz} elements. Similarly to the CSR format, the first array, \textit{ja}, contains the column index of each value per row. The second array, \textit{as}, contains the given values per row,

\begin{center}
\textit{maxnz}  =  2 
\linebreak
\linebreak

\begin{minipage}{.4\textwidth}
  \center
  \textit{as}  =   
$
\begin{pmatrix}
    & 3 & 4 & \\
    & 5 & 1 & \\
    & 11 & 12 & \\
    & 2 & 13 & \\
    & 1 & -1 & \\
\end{pmatrix}
$
\end{minipage}%
\begin{minipage}{.4\textwidth}
 \center
  \textit{ja}  =   
$
\begin{pmatrix}
    & 0 & 1 & \\
    & 1 & 2 & \\
    & 1 & 2 & \\
    & 2 & 3 & \\
    & 3 & 4 & \\
\end{pmatrix}
$
\end{minipage}
\linebreak
\end{center}

\par Following a C-like syntax, the $A . x = b$ equation, with \textit{A} following the previously described format, can be solved by,


\begin{lstlisting}
for (int i = 0; i < m; ++i) {
		double temp = 0.0;
		for (int j = 0; j < maxnz; ++j) {
				temp += as[i][j] * x[ja[i][j]];
		}
		y[i] = temp;
}	
\end{lstlisting}

 
\subsection*{Problem definition}
\addcontentsline{toc}{subsection}{Problem definition}

\par The problem consisted in analysing the performance produced by a developed kernel. This kernel was able to solve a \textbf{Sparse Matrix-Vector Multiplication}, $A.x = b$. Besides of being able to solve the previous equation, the code should be able to do it in parallel, using \textbf{OpenMP} (Open Multi-Processing) and \textbf{CUDA} (Compute Unified Device Architecture).
\par A set of matrices obtained from the University of Florida Sparse Matrix Collection\cite{sparse-matrices}, should be used to conduct the performance analysis. The files which contained the matrices were represented in the \textbf{Matrix Market} format, and they were read with software based on an existent reader. \cite{matrix-reader} 

\subsection*{Performance analysis}
\addcontentsline{toc}{subsection}{Performance analysis}

\par The performance of a given method could be measured in \textit{FLOPS}, the number of floating point operations per second. For each method, it was possible to calculate this value by the following formula,

\begin{center}
\large
\textit{FLOPS} = 
\huge
$
\frac{2 \times nz}{T}
$
\end{center}

\par In order to calculate the resulting vector in the $A.x = b$ equation, one has to execute \textbf{two} floating point operations per \textit{non-zero} value. Thus, the \textbf{numerator} expression of the previous division is obtained. The \textbf{denominator} is defined by the time, in seconds, to solve the equation. 

\subsection*{OpenMP vs CUDA}
\addcontentsline{toc}{subsection}{OpenMP vs CUDA}
OpenMP, is a technology that makes use of threads running in \textbf{C}entral \textbf{P}rocessing \textbf{U}nits. These processors, are often few compared to the number of cores present in a \textbf{G}raphics \textbf{P}rocessing \textbf{U}nits device, and the level of parallelism found in these devices is comparatively low to the level of parallelism found on GPUs. One can expect to find hundreds of cores working in parallel in a GPU. 
\par On other hand, the CUDA technology, developed by NVIDIA, offers a coding interface with a C-like syntax. Not only it allows a fast shared memory between cores within a block, but the possibility of gather data from different blocks as well. These systems allow developers to build applications capable of processing larger sets of data more quickly.\cite{cpu-gpu} 


\section*{Procedures}
\addcontentsline{toc}{section}{Procedures}
\par A parsing mechanism was developed in order to convert information in the File System into the previously referred formats. The several files were in the Matrix Market format, which the basic information is the size of the matrix, \textbf{M} rows and \textbf{N} columns, and the number of \textit{non-zero} values. These files were followed by information about each one of these values, like row, column and value itself. 
\par Both 

\subsection*{Parsing}
\addcontentsline{toc}{subsection}{Parsing}


\section*{Solution Design}
\addcontentsline{toc}{section}{Solution Design}


\section*{Results \& Discussion}
\addcontentsline{toc}{section}{Results \& Discussion}


\pagebreak
\section*{Conclusions}
\addcontentsline{toc}{section}{Conclusions}

 

%-------------------------------------------------------------------------------
% REFERENCES
%-------------------------------------------------------------------------------
\newpage
%\addcontentsline{toc}{section}{References}
\begin{thebibliography}{0}

\bibitem{sparse}
Raphael Yuster and Uri Zwick, \textit{Fast sparse matrix multiplication}, Available at: <\url{http://www.cs.tau.ac.il/~zwick/papers/sparse.pdf}> [Accessed 28 March 2017]

\bibitem{sparse-gpu}
B. Neelima1 and Prakash S. Raghavendra, April 2012,  \textit{Effective Sparse Matrix Representation for the
GPU Architectures}, Available at: <\url{https://pdfs.semanticscholar.org/2d15/dd5d0975fff797397ad31059ec097b659e00.pdf}> [Accessed 28 March 2017]

\bibitem{sparse-matrices}
University of Florida, \textit{Sparse Matrix Collection}, Available at: <\url{https://sparse.tamu.edu/}> [Accessed 28 March 2017]

\bibitem{matrix-reader}
Matrix Market, \textit{ANSI C library for Matrix Market I/O}, Available at: <\url{https://math.nist.gov/MatrixMarket/mmio-c.html}> [Accessed 28 March 2017]

\bibitem{cpu-gpu}
B. N. Manjunatha Reddy, Dr. Shanthala S., Dr. B. R. VijayaKuma, February 2017 \textit{Performance Analysis of GPU V/S CPU for Image
Processing Applications}, Available at: <\url{https://www.ijraset.com/fileserve.php?FID=6250}> [Accessed 28 March 2017]


\end{thebibliography}
\newpage

\section*{Appendices}
\addcontentsline{toc}{section}{Appendices}


\newpage

\subsection*{Source Code}
\addcontentsline{toc}{subsection}{Source Code}


\addcontentsline{toc}{subsection}{Doxygen Documentation}

\end{document}

